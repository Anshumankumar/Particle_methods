
\documentclass[12pt]{article}

\usepackage{graphicx}
\usepackage{caption}
\usepackage{subcaption}
\usepackage[font={small,it}]{caption}

\graphicspath{ {images/} }
\title {Particle Method Assignment 5 \\
Flow pass a circular Cylinder with Body cell TreeCode} 
\author {Anshuman kumar \\
Roll no.  120010036}

\begin{document}
\maketitle

\section{Introduction}
The random vortex method is used to simulate the flow across a circular cylinder. 
The simulation is done using two step. First is Advection and second is diffusion.
The diffusion is simulated using the Random walk method. The timestep used
for this assigment was 0.1 and maximum strength for vortex blob was also 
0.1.The total simulation is done for three second.

Now I am using the Multipole method for computing the vortex vortex interation . Now the algorithm is O(nlogn). 
But due to high constant factor and panel Vortex Interaction I could only simulate the cylinder with 150 panels. 

Run the cell.py in common to see the error between normal method and multipole method. It comes to be about 10e-9 for 20 coefficient. Around 500
particle mulipole method starts dominating over normal method

\section{Results}
See the flow\_pass\_cylinder.mp4 in the same directory to see the video.
The first six figure shows the feild across full cylinder. The next six shows
The Quater Cylinder for different time period.

Then we have the vortex particles 
at diffrent instant.Different colour are used to show positive and negative strenght. Finally the last plot shows the drag with time.
\begin{figure}
\centering
\begin{minipage}{.5\textwidth}
  \centering
  \includegraphics[width=.9\linewidth]{5full}
  \captionsetup{width=0.8\textwidth}
  \caption{Full cylinder V feild at 0.5 sec} 
\end{minipage}%
\begin{minipage}{.5\textwidth}
  \centering
  \includegraphics[width=.9\linewidth]{10full}
  \captionsetup{width=0.8\textwidth}
  \caption{Full cylinder V feild at 1.0 sec} 
\end{minipage}
\end{figure}

\begin{figure}
\centering
\begin{minipage}{.5\textwidth}
  \centering
  \includegraphics[width=.9\linewidth]{15full}
  \captionsetup{width=0.8\textwidth}
  \caption{Full cylinder V feild at 1.5 sec} 
\end{minipage}%
\begin{minipage}{.5\textwidth}
  \centering
  \includegraphics[width=.9\linewidth]{20full}
  \captionsetup{width=0.8\textwidth}
  \caption{Full cylinder V feild at 2.0 sec} 
\end{minipage}
\end{figure}

\begin{figure}
\centering
\begin{minipage}{.5\textwidth}
  \centering
  \includegraphics[width=.9\linewidth]{25full}
  \captionsetup{width=0.8\textwidth}
  \caption{Full cylinder V feild at 2.5 sec} 
\end{minipage}%
\begin{minipage}{.5\textwidth}
  \centering
  \includegraphics[width=.9\linewidth]{30full}
  \captionsetup{width=0.8\textwidth}
  \caption{Full cylinder V feild at 3.0 sec} 
\end{minipage}
\end{figure}


\begin{figure}
\centering
\begin{minipage}{.5\textwidth}
  \centering
  \includegraphics[width=.9\linewidth]{5quiver}
  \captionsetup{width=0.8\textwidth}
  \caption{Quater cylinder V feild at 0.5 sec} 
\end{minipage}%
\begin{minipage}{.5\textwidth}
  \centering
  \includegraphics[width=.9\linewidth]{10quiver}
  \captionsetup{width=0.8\textwidth}
  \caption{Quater cylinder V feild at 1.0 sec} 
\end{minipage}
\end{figure}

\begin{figure}
\centering
\begin{minipage}{.5\textwidth}
  \centering
  \includegraphics[width=.9\linewidth]{15quiver}
  \captionsetup{width=0.8\textwidth}
  \caption{Quater cylinder V feild at 1.5 sec} 
\end{minipage}%
\begin{minipage}{.5\textwidth}
  \centering
  \includegraphics[width=.9\linewidth]{20quiver}
  \captionsetup{width=0.8\textwidth}
  \caption{Quater cylinder V feild at 2.0 sec} 
\end{minipage}
\end{figure}

\begin{figure}
\centering
\begin{minipage}{.5\textwidth}
  \centering
  \includegraphics[width=.9\linewidth]{25quiver}
  \captionsetup{width=0.8\textwidth}
  \caption{Quater cylinder V feild at 2.5 sec} 
\end{minipage}%
\begin{minipage}{.5\textwidth}
  \centering
  \includegraphics[width=.9\linewidth]{30quiver}
  \captionsetup{width=0.8\textwidth}
  \caption{Quater cylinder V feild at 3.0 sec} 
\end{minipage}
\end{figure}

\begin{figure}
    \centering
    \includegraphics[width=.6\linewidth]{cylinder1}
    \caption{The vortex blobs at 1sec} 
\end{figure}
\begin{figure}
    \centering
    \includegraphics[width=.6\linewidth]{cylinder2}
    \caption{The vortex blobs at 1sec} 
\end{figure}
\begin{figure}
    \centering
    \includegraphics[width=.6\linewidth]{cylinder3}
    \caption{The vortex blobs at 1sec} 
\end{figure}





\begin{figure}
    \centering
    \includegraphics[width=.9\linewidth]{drag}
    \caption{The drag variation with time} 
\end{figure}

\end{document}
